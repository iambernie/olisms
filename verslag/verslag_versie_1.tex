\documentclass[11pt,a4paper]{article}

\usepackage{mathtools}
\usepackage{graphicx}
\usepackage{bm}
\usepackage{calligra}
\usepackage{wrapfig}
\usepackage{comment}
\usepackage{subcaption}
\usepackage{color}

%\DeclareMathAlphabet{\mathcalligra}{T1}{calligra}{m}{n}
%\DeclareFontShape{T1}{calligra}{m}{n}{<->s*[2.2]callig15}{}

\setlength{\parindent}{0pt}


\DeclarePairedDelimiter{\avg}{\langle}{\rangle}

\title{The Lau/Ostoji\'c Ising Model simulation}
\date{}
\author{Bernie Lau and Oliver Ostoji\'c}

\begin{document}

\maketitle

\section{Introduction}
{\color{red} Some introductory words here, not too many words, not too much physics.}
\subsection{A bit of theory}
In effect, we are investigating the famous Ising Hamiltonian without external field;

\begin{equation}\label{eq:Hamiltonian}
    H = -\beta J \sum_{\avg*{i,j}} \vec{S}_i \cdot \vec{S}_j
\end{equation}

The $\avg{i,j}$ indicates a sum over nearest neighbors, $\beta = 1/k_bT$ is the inverse temperature and $J$ is a
 coupling constant indicating the strength of the interaction between neigbouring spins. In the whole experiment,
 we set $J = k_b = 1$ which mesns we measure the (inverse) temperature in arbitrary units. This is justified by the
 fact that we are mostly interested in the qualitative behaviour of the Hamiltonian. Also, since the lattice sizes we
 will be simulating are about 20 orders of magnitude smaller than physical systems, the notion of temperature is not 
 as clearly defined anyway. \\
 Two algorithms are implemented in this simulation.

\subsection{Our approach}


\section{Thermalization and autocorrelation for the 40x40 lattice}


In this section we present some results of a simulation of a 40x40 two dimensional Ising model, using a Metropolis algorithm.
 We present plots of energy and magnetization as a function of time and report the thermalization times found for each plot.
 We also plot autocorrelation functions for both enery and magnetization
 for five values of the inverse temperature $\beta$ between $\beta = 0.3$ and $\beta = 0.5$. 
 Both high temperature (fully disordered) and temperature (all spins aligned) initial conditions are simulated.
 Time is measured in Monte-Carlo steps per spin for all simulations. Each simlutation "takes" $32\cdot 10^6$ 
 Monte-Carlo steps.


\subsection{Energy functions}
To find the thermalization times, we plot the energy as a function of time (measured in Monte-Carlo steps per site)
 and fit a general exponential curve and read off the characteristic time $\tau$:

\begin{equation}\label{eq:exp_decay}
    f(t)=Ae^{t/\tau} + C
\end{equation}

After a single characteristic time $\tau$, the value of the energy will be at about $37\%$ of its initial value and
 the system will not yet be thermalized. We therefore define the
 thermalization time as five times the characteristic time, just to be safe:
 
\begin{equation*}
    t_{therm} = 5\tau
\end{equation*}

Figure \ref{fig:Evt} shows the results of this procedure for the values $\beta = 0.3$; $\beta = 0.35$;
 $\beta = 0.4$; $\beta = 0.45$ and $\beta = 0.5$. Included are the acquired values of $t_{therm}$. Both 
 high temperature and low temperature initial conditions are covered.
\\
\\
{\color{red}REVIZE SECTION AFTER ADDING PLOTS AND WRITING INTRO}

\begin{figure}[h!]
\centering
\begin{subfigure}{.5\textwidth}
  \centering
  \includegraphics[width=.4\linewidth]{Boltzmann.jpg}
  \caption{}
  \label{fig:Evt_highT}
\end{subfigure}%
\begin{subfigure}{.5\textwidth}
  \centering
  \includegraphics[width=.4\linewidth]{Boltzmann.jpg}
  \caption{}
  \label{fig:Evt_lowT}
\end{subfigure}
\caption{Energy vs ierations per spin for the values $\beta = 0.3$; $\beta = 0.35$;
         $\beta = 0.4$; $\beta = 0.45$ and $\beta = 0.5$. a) High temperature starting condition. 
         {\color{red}To add: the (calculated) values of the thermalization times per $\beta$}.
         b) Low temperature starting condition.
         {\color{red}To add: the (calculated) values of the thermalization times per $\beta$)}}
\label{fig:Evt}
\end{figure}


\subsection{Magnetization functions}
As we are still looking for thermalization times, just now for magnetization, we use the same procedure as in
 the previous section. 

Figure \ref{fig:Mvt} shows plots of magnetization versus time for the values $\beta = 0.3$; $\beta = 0.35$;
 $\beta = 0.4$; $\beta = 0.45$ and $\beta = 0.5$ for high and low temperature initial conditions. 


\begin{figure}[h!]
\centering
\begin{subfigure}{.5\textwidth}
  \centering
  \includegraphics[width=.4\linewidth]{Boltzmann.jpg}
  \caption{}
  \label{fig:Mvt_highT}
\end{subfigure}%
\begin{subfigure}{.5\textwidth}
  \centering
  \includegraphics[width=.4\linewidth]{Boltzmann.jpg}
  \caption{}
  \label{fig:Mvt_lowT}
\end{subfigure}
\caption{Magnetization vs ierations per spin for the values $\beta = 0.3$; $\beta = 0.35$;
         $\beta = 0.4$; $\beta = 0.45$ and $\beta = 0.5$. a) High temperature starting condition
         {\color{red}To add: the (calculated) values of the thermalization times per $\beta$}.
         b) Low temperature starting condition.
         {\color{red}To add: the (calculated) values of the thermalization times per $\beta$}.}
\label{fig:Mvt}
\end{figure}


\subsection{Autocorrelation functions}
In addition to the previous results, the autocorrelation times for both energy and magnetization can be
 extracted from the simulations. We implement the formulas

\begin{equation*}
    c_e(\Delta t) = \avg{(E(t+\Delta t) - E_{avg})\cdot (E(t)-E_{avg})}_t
\end{equation*}

and

\begin{equation*}
    c_m(\Delta t) = \avg{(M(t+\Delta t) - M_{avg})\cdot (M(t)-M_{avg})}_t
\end{equation*}


where, $E_{avg}$ and $M_{avg}$ indicate averaging over the \textit{last} million values of the energy and magnetization
 respectively. This in contrast to the averaging indicated by the $\avg{}_t$, which means we average over the number of
 iterations. We again fit the general exponential decay given by equation \ref{eq:exp_decay}, and report the characteristic
 time, which is by definition the autocorrelation time. The results are given below (Figure \ref{fig:autocorr})
 for each simulation, i.e for the same values of $\beta$ as listed above.
 
\begin{figure}[h!]
\centering
\begin{subfigure}{.5\textwidth}
  \centering
  \includegraphics[width=.4\linewidth]{Boltzmann.jpg}
  \caption{}
  \label{fig:autocorr_energy}
\end{subfigure}%
\begin{subfigure}{.5\textwidth}
  \centering
  \includegraphics[width=.4\linewidth]{Boltzmann.jpg}
  \caption{}
  \label{fig:autocorr_mag}
\end{subfigure}
\caption{Autocorrelation times vs $\Delta t$ for the values $\beta = 0.3$; $\beta = 0.35$;
         $\beta = 0.4$; $\beta = 0.45$ and $\beta = 0.5$. a) Energy autocorrelation function. 
         b) Magnetization autocorrelation function.}
\label{fig:autocorr}
\end{figure}


\section{Magnetization and magnetic suscepibility with metropolis}

\section{The Wolff algorithm}

\section{The 3D Ising model}
\end{document}
