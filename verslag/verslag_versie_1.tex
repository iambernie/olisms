\documentclass[11pt,a4paper]{article}

\usepackage{mathtools}
\usepackage{graphicx}
\usepackage{bm}
\usepackage{calligra}
\usepackage{wrapfig}
\usepackage{comment}
\usepackage{subcaption}
\usepackage{color}

%\DeclareMathAlphabet{\mathcalligra}{T1}{calligra}{m}{n}
%\DeclareFontShape{T1}{calligra}{m}{n}{<->s*[2.2]callig15}{}

\setlength{\parindent}{0pt}


\DeclarePairedDelimiter{\avg}{\langle}{\rangle}

\title{The Lau/Ostoji\'c Ising model simulation}
\date{}
\author{Bernie Lau and Oliver Ostoji\'c}

\begin{document}

\maketitle

\section{Introduction}

This project is in essence an investigation of the Ising Hamiltonian with no external field;\footnote{As always,
$\avg{i,j}$ indicates a sum over nearest neighbors}

\begin{equation}\label{eq:Hamiltonian}
    H = -J \sum_{\avg*{i,j}} s_i s_j
\end{equation}

As this apparently simple Hamiltonian is known to result in very interesting and rich dynamical behavior on a 
 macroscopic scale, we endeavor to find, through computational methods, some interesting thermodynamical quantities
 and qualitative behaviors. Of primary interest is the search for signs of a phase transistion, which the two and three
 dimensional Ising model is known to exhibit.

 In order to be less vague, we shall from now on talk about the Ising Hamiltonian
 in the context of ferromagnetism, where the quantites $s_i$ are associated with electron spins which can point in one of two
 directions. We will also set the coupling energy and the Boltzmann canstant equal to one, 
 $J = k_B = 1$, for all simulations. As a consequence, the inverse temperature $\beta = 1/k_BT$ is measured in unphysical
 units. This is justified by the
 fact that  we are primarily interested in the qualitative behavior of the system and all actual numbers in the following
 simulations can still be compared to theoretical calculations and similar simulations.

 Phase transitions are typically characterized by rapid changes in macroscopic quantities near the transition temperature, which
 is called the critical temperature, $T_c$. The macroscopic quantities we will be investigating are the net magnetization,
 $M = \sum_i s_i$, the magnetic susceptibility (a measure for how the magnetization reacts to external magnetic fields) and
 the correlation length (a measure for the length scale at which the spins influence each other). These quantities and their
 connection to phase transitions will be made more concrete in the relevant sections. 

 The actual methods we employ are two Monte Carlo type algorithms, with most simulations carried out using a 
 Metropolis algorithm and a couple of measurements using a Wolff algorithm. It seems appropriate therefore, to summarize the
 principles of those algorithms here, in a minimalistic fashion.

\subsection{Metropolis algorithm}
As the focus of this assignment is on the \textit{implementation} of the Metropolis algorithm rather than its theoretical
 aspects, we have chosen to omit the technical details and just outline the procedure used in its implementation here.

 Given a
 lattice with a spin variable $s_i \in \{-1,1\}$ on each lattice point $i$, we define a "state" to be a particular
 configuration of all the ${s_i}$. The energy $E_\mu$ of the system in a state $\mu$ is determined by the Hamiltonian
 (Equation \ref{eq:Hamiltonian}). Time evolution of the system is achieved by randomly choosing a single site $j$ and determining
 the energy $E_\nu$ of the system if the spin on that site were flipped, $s_j \rightarrow -s_j$. Weather or not we actually flip
 the spin is determined by the the number $A(\mu \rightarrow \nu)$ given by;


\begin{equation}\label{eq:A-ratio}
    A(\mu \rightarrow \nu) = \begin{cases}
        e^{-\beta (E_\nu - E_\mu)} & \mbox{if} \,\,  E_\nu - E_\mu > 0 \\
        1 & \mbox{else}
    \end{cases}
\end{equation}

Once the spin flip is accepted or rejected, the process is repeated and each iteration is called a "Monte Carlo step". It is actually
 this particular definiton of the so called
 "acceptance ratio", $A(\mu \rightarrow \nu)$ that makes the algorihm a Metropolis algorithm,
 the rest of the procedure is common to all single spin-flip dynamics Monte Carlo algorithms. The conditions
 of detailed balance and ergodicity, which ensure that the
 equilibrium state of the system will behave according to Boltzmann statistics and that each state can be reached from each other state, are
 statisfied by this acceptance ratio, as must be the case with all Monte Carlo algorithms.


\subsection{Wolff algorithm}
An example of a cluster-flipping algorithm, the Wolff algorithm works by iteratively picking out
 an entire cluster of spins pointing in the same direction and then flipping them all at the same time. The actual physics content of this 
 procedure is contained in the generation of the cluster, which is done based on a temperature dependent probability chosen such that the
 conditions of detailed balance and ergodicity are satisfied.
 More specifically, the procedure which is iterated is as followsi;
 \begin{enumerate}
 \item Pick a spin at random and look at all its neighbors.
 \item If a neighbor is pointing in the same direction as the current one, add it to the cluster with a probability $P_{add}$ given by

\begin{equation}\label{eq:A-ratio}
    P_{add} = 1 - e^{-2\beta J}
\end{equation}
 

 \item For each spin that was added, look at all \textit{its} neighbors and add them with the same probability. Do this until no spins are left
 in the cluster whose neighbors have not been considered for addition.
 \item Flip the cluster

 \end{enumerate}

The Wolff algorithm, although less intuitive than Metropolis has the advantage of being much faster near the critical temperature, thereby
 allowing us to obtain more accurate results in the region of interest.

\section{Thermalization and autocorrelation for the 40x40 lattice}


In this section we present some results of a simulation of a 40x40 two dimensional Ising model, using a Metropolis algorithm.
 We present plots of energy and magnetization as a function of time and report the thermalization times found for each plot.
 We also plot autocorrelation functions for both enery and magnetization
 for five values of the inverse temperature $\beta$ between $\beta = 0.3$ and $\beta = 0.5$. 
 Both high temperature (fully disordered) and temperature (all spins aligned) initial conditions are simulated.
 Time is measured in Monte-Carlo steps per spin for all simulations. Each simlutation "takes" $32\cdot 10^6$ 
 Monte-Carlo steps.


\subsection{Energy functions}
To find the thermalization times, we plot the energy as a function of time (measured in Monte-Carlo steps per site)
 and fit a general exponential curve and read off the characteristic time $\tau$:

\begin{equation}\label{eq:exp_decay}
    f(t)=Ae^{t/\tau} + C
\end{equation}

After a single characteristic time $\tau$, the value of the energy will be at about $37\%$ of its initial value and
 the system will not yet be thermalized. We therefore define the
 thermalization time as five times the characteristic time, just to be safe:
 
\begin{equation*}
    t_{therm} = 5\tau
\end{equation*}

Figure \ref{fig:Evt} shows the results of this procedure for the values $\beta = 0.3$; $\beta = 0.35$;
 $\beta = 0.4$; $\beta = 0.45$ and $\beta = 0.5$. Included are the acquired values of $t_{therm}$. Both 
 high temperature and low temperature initial conditions are covered.
\\
\\
{\color{red}REVIZE SECTION AFTER ADDING PLOTS AND WRITING INTRO}

\begin{figure}[h!]
\centering
\begin{subfigure}{.5\textwidth}
  \centering
  \includegraphics[width=\linewidth]{metro_40x40_iter_32M_unaligned_E_vs_time.png}
  \caption{}
  \label{fig:Evt_highT}
\end{subfigure}%
\begin{subfigure}{.5\textwidth}
  \centering
  \includegraphics[width=\linewidth]{metro_40x40_iter_32M_aligned_E_vs_time.png}
  \caption{}
  \label{fig:Evt_lowT}
\end{subfigure}
\caption{Energy vs ierations per spin for the values $\beta = 0.3$; $\beta = 0.35$;
         $\beta = 0.4$; $\beta = 0.45$ and $\beta = 0.5$. a) High temperature starting condition. 
         {\color{red}To add: the (calculated) values of the thermalization times per $\beta$}.
         b) Low temperature starting condition.
         {\color{red}To add: the (calculated) values of the thermalization times per $\beta$)}}
\label{fig:Evt}
\end{figure}


\subsection{Magnetization functions}
As we are still looking for thermalization times, just now for magnetization, we use the same procedure as in
 the previous section. 

Figure \ref{fig:Mvt} shows plots of magnetization versus time for the values $\beta = 0.3$; $\beta = 0.35$;
 $\beta = 0.4$; $\beta = 0.45$ and $\beta = 0.5$ for high and low temperature initial conditions. 


\begin{figure}[h!]
\centering
\begin{subfigure}{.5\textwidth}
  \centering
  \includegraphics[width=\linewidth]{metro_40x40_iter_32M_unaligned_net_M_vs_time.png}
  \caption{}
  \label{fig:Mvt_highT}
\end{subfigure}%
\begin{subfigure}{.5\textwidth}
  \centering
  \includegraphics[width=\linewidth]{metro_40x40_iter_32M_aligned_net_M_vs_time.png}
  \caption{}
  \label{fig:Mvt_lowT}
\end{subfigure}
\caption{Magnetization vs ierations per spin for the values $\beta = 0.3$; $\beta = 0.35$;
         $\beta = 0.4$; $\beta = 0.45$ and $\beta = 0.5$. a) High temperature starting condition
         {\color{red}To add: the (calculated) values of the thermalization times per $\beta$}.
         b) Low temperature starting condition.
         {\color{red}To add: the (calculated) values of the thermalization times per $\beta$}.}
\label{fig:Mvt}
\end{figure}


\subsection{Autocorrelation functions}
In addition to the previous results, the autocorrelation times for both energy and magnetization can be
 extracted from the simulations. We implement the formulas

\begin{equation*}
    c_e(\Delta t) = \avg{(E(t+\Delta t) - E_{avg})\cdot (E(t)-E_{avg})}_t
\end{equation*}

and

\begin{equation*}
    c_m(\Delta t) = \avg{(M(t+\Delta t) - M_{avg})\cdot (M(t)-M_{avg})}_t
\end{equation*}


where, $E_{avg}$ and $M_{avg}$ indicate averaging over the \textit{last} million values of the energy and magnetization
 respectively. This in contrast to the averaging indicated by the $\avg{}_t$, which means we average over the number of
 iterations. We again fit the general exponential decay given by equation \ref{eq:exp_decay}, and report the characteristic
 time, which is by definition the autocorrelation time. The results are given below (Figure \ref{fig:autocorr})
 for each simulation, i.e for the same values of $\beta$ as listed above.
 
\begin{figure}[h!]
\centering
\begin{subfigure}{.5\textwidth}
  \centering
  \includegraphics[width=.4\linewidth]{Boltzmann.jpg}
  \caption{}
  \label{fig:autocorr_energy}
\end{subfigure}%
\begin{subfigure}{.5\textwidth}
  \centering
  \includegraphics[width=.4\linewidth]{Boltzmann.jpg}
  \caption{}
  \label{fig:autocorr_mag}
\end{subfigure}
\caption{Autocorrelation times vs $\Delta t$ for the values $\beta = 0.3$; $\beta = 0.35$;
         $\beta = 0.4$; $\beta = 0.45$ and $\beta = 0.5$. a) Energy autocorrelation function. 
         b) Magnetization autocorrelation function.}
\label{fig:autocorr}
\end{figure}


\section{Magnetization and magnetic suscepibility with metropolis}

\section{The Wolff}

\section{The 3D Ising model}
\end{document}
